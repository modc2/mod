\documentclass{article}
\usepackage[utf8]{inputenc}
\usepackage{hyperref}
\usepackage{listings}
\usepackage{xcolor}

\title{Agent Module: How It Works}
\author{Technical Documentation}
\date{\today}

\definecolor{codegreen}{rgb}{0,0.6,0}
\definecolor{codegray}{rgb}{0.5,0.5,0.5}
\definecolor{codepurple}{rgb}{0.58,0,0.82}
\definecolor{backcolour}{rgb}{0.95,0.95,0.92}

\lstdefinestyle{mystyle}{
    backgroundcolor=\color{backcolour},
    commentstyle=\color{codegreen},
    keywordstyle=\color{magenta},
    numberstyle=\tiny\color{codegray},
    stringstyle=\color{codepurple},
    basicstyle=\ttfamily\footnotesize,
    breakatwhitespace=false,
    breaklines=true,
    captionpos=b,
    keepspaces=true,
    numbers=left,
    numbersep=5pt,
    showspaces=false,
    showstringspaces=false,
    showtabs=false,
    tabsize=2
}

\lstset{style=mystyle}

\begin{document}

\maketitle
\tableofcontents
\newpage

\section{Introduction}

The Agent Module is an advanced code generation and editing toolkit powered by Large Language Models (LLMs). It provides a powerful interface for generating, editing, and managing code through natural language prompts.

\section{Architecture Overview}

\subsection{Core Components}

The Agent Module consists of several interconnected components:

\begin{itemize}
    \item \textbf{Agent}: Main orchestrator for code generation and editing
    \item \textbf{Edit}: Specialized module for modifying existing code
    \item \textbf{Select}: Semantic file selection tool
    \item \textbf{Memory}: Context management system (short-term and long-term)
    \item \textbf{Toolbox}: Helper and guide system
\end{itemize}

\subsection{Workflow}

The Agent Module operates through the following workflow:

\begin{enumerate}
    \item \textbf{Input Processing}: User provides natural language query
    \item \textbf{Context Building}: Memory and relevant files are loaded
    \item \textbf{Tool Selection}: Agent determines which tools to use
    \item \textbf{Execution}: Tools are executed in sequence
    \item \textbf{Output Generation}: Results are formatted and returned
\end{enumerate}

\section{How It Works}

\subsection{Code Generation}

When generating code, the Agent:

\begin{lstlisting}[language=Python]
import mod as c

# Initialize the dev module
dev = c.mod('dev')()

# Generate code from a prompt
dev.forward("Create a Python function that calculates Fibonacci numbers")
\end{lstlisting}

\textbf{Process:}
\begin{enumerate}
    \item Parse the natural language prompt
    \item Query the LLM with context and requirements
    \item Generate code based on best practices
    \item Return formatted code output
\end{enumerate}

\subsection{Code Editing}

For editing existing code:

\begin{lstlisting}[language=Python]
dev.forward("Add error handling to the function", to="./path/to/file.py")
\end{lstlisting}

\textbf{Process:}
\begin{enumerate}
    \item Load the target file content
    \item Analyze the existing code structure
    \item Apply modifications based on the prompt
    \item Preview changes for user confirmation
    \item Write updated content to file
\end{enumerate}

\subsection{File Selection}

Semantic file selection uses embeddings:

\begin{lstlisting}[language=Python]
select = c.mod('tool.select')()
files = c.files("./project")
auth_files = select.forward(
    options=files,
    query="files related to authentication"
)
\end{lstlisting}

\textbf{Process:}
\begin{enumerate}
    \item Generate embeddings for all files
    \item Generate embedding for the query
    \item Calculate semantic similarity
    \item Return ranked list of relevant files
\end{enumerate}

\subsection{Memory Management}

The memory system maintains context:

\begin{lstlisting}[language=Python]
memory = c.mod('tool.memory')()

# Short-term memory (session-based)
memory.add_short_term("user_preference", {"theme": "dark"})

# Long-term memory (persistent)
memory.add_long_term("project_requirements", {
    "name": "AI Assistant",
    "features": ["code generation", "memory management"]
})
\end{lstlisting}

\textbf{Memory Types:}
\begin{itemize}
    \item \textbf{Short-term}: Session-specific context, cleared after session
    \item \textbf{Long-term}: Persistent storage across sessions
\end{itemize}

\subsection{Function Calling}

Dynamic function integration:

\begin{lstlisting}[language=Python]
dev.forward("Document these functions: @/get_text ./utils/helpers.py")
\end{lstlisting}

\textbf{Process:}
\begin{enumerate}
    \item Parse function call syntax (@/function\_name)
    \item Execute the referenced function
    \item Inject results into the prompt context
    \item Process the enriched prompt
\end{enumerate}

\section{Tool System}

\subsection{Available Tools}

The Agent has access to multiple tools:

\begin{itemize}
    \item \texttt{create\_file}: Create new files with content
    \item \texttt{edit\_file}: Modify existing files
    \item \texttt{rm\_file}: Remove files
    \item \texttt{cmd}: Execute shell commands
    \item \texttt{select\_files}: Find relevant files
    \item \texttt{summarize}: Generate content summaries
    \item \texttt{web\_scraper}: Search and scrape web content
\end{itemize}

\subsection{Tool Execution}

Tools are executed through a standardized interface:

\begin{enumerate}
    \item Agent receives user query
    \item Determines required tools and parameters
    \item Executes tools in sequence
    \item Aggregates results
    \item Returns formatted output
\end{enumerate}

\section{LLM Integration}

\subsection{Model Configuration}

The Agent supports multiple LLM providers:

\begin{itemize}
    \item Default: \texttt{anthropic/claude-3.7-sonnet}
    \item Configurable parameters: temperature, max\_tokens, model selection
\end{itemize}

\subsection{Prompt Engineering}

The Agent uses sophisticated prompt engineering:

\begin{itemize}
    \item Context injection from memory
    \item Tool schema descriptions
    \item Few-shot examples
    \item Chain-of-thought reasoning
\end{itemize}

\section{Advanced Features}

\subsection{Interactive Workflow}

Changes can be previewed before application:

\begin{enumerate}
    \item Generate proposed changes
    \item Display diff to user
    \item Request confirmation
    \item Apply or reject changes
\end{enumerate}

\subsection{Error Handling}

Robust error handling throughout:

\begin{itemize}
    \item File operation errors
    \item LLM API failures
    \item Invalid tool parameters
    \item Graceful degradation
\end{itemize}

\section{Use Cases}

\subsection{Rapid Prototyping}

Quickly generate boilerplate code and project structures.

\subsection{Code Refactoring}

Modernize and improve existing codebases with natural language instructions.

\subsection{Documentation}

Automatically generate documentation from code.

\subsection{Learning Assistant}

Help developers understand and work with unfamiliar codebases.

\section{Conclusion}

The Agent Module provides a powerful, flexible interface for AI-assisted software development. By combining LLM capabilities with a robust tool system and memory management, it enables developers to work more efficiently and effectively.

\end{document}